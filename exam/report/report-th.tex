\documentclass[a4paper, 11pt]{article}

\usepackage[utf8x]{inputenc}
\usepackage[T1]{fontenc}
\usepackage{ucs}
\usepackage[english]{babel}
\usepackage{mathtools, amsmath, amsfonts}
\usepackage{fancyhdr}
\usepackage[parfill]{parskip}
\usepackage{graphicx}
\usepackage{palatino,newtxmath}
\usepackage{float}
\usepackage[font={small,it}]{caption}
\usepackage{fixltx2e}
\usepackage{zed-csp}
\usepackage[scaled]{helvet}
\usepackage{fullpage}

\linespread{1.05}
\pagestyle{fancyplain}
\fancyhead{}
\fancyfoot[L]{}
\fancyfoot[C]{}
\fancyfoot[R]{\thepage}
\renewcommand{\headrulewidth}{0pt}
\renewcommand{\footrulewidth}{0pt}
\setlength{\headheight}{13.6pt}

\setcounter{secnumdepth}{0}
\widowpenalty=1000
\clubpenalty=1000

\newcommand{\horrule}[1]{\rule{\linewidth}{#1}}

% CSP shorthands
\def\Prod{\textit{PROD}}
\def\Prodi{\textit{PRODI}}
\def\Prodr{\textit{PRODR}}
\def\Prodnx{\textit{PRODN}_x}
\def\Proda{\textit{PRODA}_x}
\def\Prodb{\textit{PRODB}_y}
\def\Prodc{\textit{PRODC}_{x,y}}
\def\Qual{\textit{QUAL}}
\def\Mill{\textit{MILL}}
\def\Stop{\textit{STOP}}
\def\inc{\textsf{in}}
\def\chk{\textsf{chk}}
\def\rej{\textsf{rej}}
\def\outc{\textsf{out}}
\newcommand{\law}[2]{[#1\;\text{#2}]}
\def\lcd{[\text{LCD}]}
\newcommand{\Tmp}[2]{\textit{TMP#1}_{#2}}
\newcommand{\Prodn}[1]{\textit{PRODN}_{#1}}
\newcommand{\chan}[1]{|\!\!#1\!\!|}
\def\after{\;/\;}
\def\Refusals{\textit{refusals}}
\def\Interleaves{\;\textit{interleaves}\;}

\title{ 
\normalfont \normalsize 
\textsc{University of Copenhagen} \\ [25pt]
\horrule{0.5pt} \\[0.4cm]
\huge XMP: Exam \\ \Large - Theory \\
\horrule{2pt} \\[0.5cm]
}

\author{Jens Fredskov (chw752)}

\begin{document}
\maketitle
\pagebreak

\section{Problem 1} % (fold)
\label{sec:problem_1}

\subsection{a)} % (fold)

We first consider our given processes to eliminate the general choice in $\Prod$.
\begin{align*}
    \Prod
    &= (\inc ? x \then \chk ! (x-4) \then \Prod) \extchoice
       (\rej ? y \then \chk ! (y-2) \then \Prod) \\
    &= (\inc ? x \then \chk ! (x-4) \then \Prod) \barchoice
       (\rej ? y \then \chk ! (y-2) \then \Prod) & \law{3.3.1}{L5} \\ \\
    \Qual
    &= \chk ? z \then ((\outc ! z \then \Qual) \lhd ok(z) \rhd
                       (\rej ! z \then \Qual))
\end{align*}
We now define an auxiliary process $\Tmp{0}{x}$
\begin{align*}
    \Tmp{0}{x}
    &= (\chk ! x \then \Prod) \parallel \Qual \\
    &= \chk . x \then (\Prod \parallel ((c ! z \then \Qual) \lhd ok(z) \rhd (\rej ! z \then \Qual))) & \law{2.3.1}{L4A} \\
    &= \chk . x \then ((\Prod \parallel (c ! z \then \Qual)) \lhd ok(z) \rhd (\Prod \parallel (\rej ! z \then \Qual))) & \lcd
    \intertext{With the sets $A=\set{\chan{\inc},\chan{\rej}}, B=\set{\outc . x}, C=\set{\chan{\inc}, \outc . x}$, the left hand side of the \textit{if-then}-clause becomes}
    LHS
    &= \Prod \parallel (\outc ! x \then \Qual) \\
    &= (\inc ? z \then ((\chk ! (z-4) \then \Prod) \parallel (\outc ! x \then \Qual)) \\
    &\quad \barchoice \outc ! x \then (\Prod \parallel \Qual)) & \law{2.3.1}{L7} \\
    &= (\inc ? z \then \outc ! x \then ((\chk ! (z-4) \then \Prod) \parallel \Qual) \\
    &\quad \barchoice \outc ! x \then \inc ? z \then ((\chk ! (z-4) \then \Prod) \parallel \Qual)) & \law{2.3.1}{L5A/B} \\
    &= (\inc ? z \then \outc ! x \then \Tmp{0}{z-4} \barchoice \outc ! x \then \inc ? z \then \Tmp{0}{z-4})
    \intertext{With the sets $A=\set{\chan{\inc},\chan{\rej}}, B=\set{\rej . x}, C=\set{\chan{\inc}, \rej . x}$, the right hand side becomes}
    RHS
    &= \Prod \parallel (\rej ! x \then \Qual) \\
    &= (\inc ? z \then ((\chk ! (z-4) \Prod) \parallel (\rej ! x \then \Qual)) \\
    &\quad \barchoice \rej . x \then ((\chk ! (x-2) \then \Prod) \parallel \Qual)) & \law{2.3.1}{L7} \\
    &= (\inc ? z \then \Stop \barchoice \rej . x \then \Tmp{0}{x-2})
    \intertext{We then put the side together again}
    \Tmp{0}{x}
    &= \chk . x \then ((\inc ? z \then \outc ! x \then \Tmp{0}{z-4} \barchoice \outc ! x \then \inc ? z \then \Tmp{0}{z-4}) \\
    &\quad \lhd ok(x) \rhd (\inc ? z \then \Stop \barchoice \rej . x \then \Tmp{0}{x-2}))
\end{align*}
We now define a new process which is the same as before, but with $CR$ concealed.
\begin{align*}
    \Tmp{1}{x}
    &= \Tmp{0}{x} \hide CR \\
    &= (\chk . x \then ((\inc ? z \then \outc ! x \then \Tmp{0}{z-4} \barchoice \outc ! x \then \inc ? z \then \Tmp{0}{z-4}) \\
    &\quad \lhd ok(x) \rhd (\inc ? z \then \Stop \barchoice \rej . x \then \Tmp{0}{x-2}))) \hide CR \\
    &= ((\inc ? z \then \outc ! x \then \Tmp{0}{z-4} \barchoice \outc ! x \then \inc ? z \then \Tmp{0}{z-4}) \\
    &\quad \lhd ok(x) \rhd (\inc ? z \then \Stop \barchoice \rej . x \then \Tmp{0}{x-2})) \hide CR & \law{3.5.1}{L5} \\
    &= (((\inc ? z \then \outc ! x \then \Tmp{0}{z-4} \barchoice \outc ! x \then \inc ? z \then \Tmp{0}{z-4}) \\
    &\quad \hide CR) \lhd ok(x) \rhd ((\inc ? z \then \Stop \barchoice \rej . x \then \Tmp{0}{x-2}) \hide CR))  & \lcd
    \intertext{Then twice with sets $B=\set{\chan{\inc}, \outc . x}, C=\set{\chan{\chk}, \chan{\rej}}$}
    &= ((\inc ? z \then \outc ! x \then \Tmp{1}{z-4} \barchoice \outc ! x \then \inc ? z \then \Tmp{1}{z-4}) \\
    &\quad \lhd ok(x) \rhd ((\inc ? z \then \Stop \barchoice \rej . x \then \Tmp{0}{x-2}) \hide CR)) & \law{3.5.1}{L8}
    \intertext{Finally with sets $B=\set{\chan{\inc}, \rej . x}, C=\set{\chan{\chk}, \chan{\rej}}$}
    &= ((\inc ? z \then \outc ! x \then \Tmp{1}{z-4} \barchoice \outc ! x \then \inc ? z \then \Tmp{1}{z-4}) \\
    &\quad \lhd ok(x) \rhd (\Tmp{1}{x-2} \intchoice (\Tmp{1}{x-2} \extchoice (\inc ? z \then \Stop)))) &\law{3.5.1}{L10} \\ \\
    \intertext{To eliminate the general choice after hiding $CR$ we define a new process}
    \Tmp{2}{x}
    &= \Tmp{1}{x} \extchoice (\inc ? z \then \Stop) \\
    &= ((\inc ? z \then \outc ! x \then \Tmp{1}{z-4} \barchoice \outc ! x \then \inc ? z \then \Tmp{1}{z-4}) \\
    &\quad \lhd ok(x) \rhd (\Tmp{1}{x-2} \intchoice (\Tmp{1}{x-2} \extchoice (\inc ? z \then \Stop)))) \extchoice (\inc ? z \then \Stop) \\
    &= (((\inc ? z \then \outc ! x \then \Tmp{1}{z-4} \barchoice \outc ! x \then \inc ? z \then \Tmp{1}{z-4}) \extchoice (\inc ? z \then \Stop)) \\
    &\quad \lhd ok(x) \rhd ((\Tmp{1}{x-2} \intchoice (\Tmp{1}{x-2} \extchoice (\inc ? z \then \Stop))) \extchoice (\inc ? z \then \Stop))) & \lcd
    \intertext{The left hand side of the \textit{if-then}-clause then becomes}
    LHS
    &= (\inc ? z \then \outc ! x \then \Tmp{1}{z-4} \barchoice \outc ! x \then \inc ? z \then \Tmp{1}{z-4}) \extchoice (\inc ? z \then \Stop) \\
    &= (\inc ? z \then ((\outc ! x \then \Tmp{1}{z-4}) \intchoice \Stop) \barchoice \outc ! x \then \inc ? z \then \Tmp{1}{z-4}) & \law{3.3.1}{L5}
    \intertext{And the right hand side becomes}
    RHS
    &= (\Tmp{1}{x-2} \intchoice (\Tmp{1}{x-2} \extchoice (\inc ? z \then \Stop))) \extchoice (\inc ? z \then \Stop) \\
    &= (\Tmp{1}{x-2} \extchoice (\inc ? z \then \Stop)) \intchoice ((\Tmp{1}{x-2} \extchoice (\inc ? z \then \Stop)) \extchoice (\inc ? z \then \Stop)) & \law{3.3.1}{L6}\\
    &= \Tmp{2}{x-2} \intchoice \Tmp{2}{x-2} & \law{3.3.1}{L1-L3} \\
    &= \Tmp{2}{x-2} & \law{3.2.1}{L1}
    \intertext{We then put the side together again}
    \Tmp{2}{x}
    &= ((\inc ? z \then ((\outc ! x \then \Tmp{1}{z-4}) \intchoice \Stop) \barchoice \outc ! x \then \inc ? z \then \Tmp{1}{z-4}) \\
    &\quad \lhd ok(x) \rhd \Tmp{2}{x-2})
    \intertext{Finally we can use this in the definition of the earlier process}
    \Tmp{1}{x}
    &= ((\inc ? z \then \outc ! x \then \Tmp{1}{z-4} \barchoice \outc ! x \then \inc ? z \then \Tmp{1}{z-4}) \\
    &\quad \lhd ok(x) \rhd (\Tmp{1}{x-2} \intchoice \Tmp{2}{x-2})) \\
\end{align*}
We are now ready to find an equivalent definition for $\Mill$ using the subprocesses we have created.
\begin{align*}
    \Mill
    &= (\Prod \parallel \Qual) \hide CR \\
    &= (\inc ? x \then ((\chk ! (x-4) \then \Prod) \parallel \Qual)) \hide CR & \law{2.3.1}{L5A} \\
    &= \inc ? x \then (\Tmp{0}{x-4} \hide CR) & \law{3.5.1}{L5} \\
    &= \inc ? x \then \Tmp{1}{x-4}
\end{align*}

% subsection a_ (end)

\subsection{b)} % (fold)
One trace which could make $\Mill$ deadlock is $s = \trace{\inc . 106, \inc . 106}$. We first show that $s \in \Traces(\Mill)$.
\begin{align*}
    \Traces(\Mill)
    &= \Traces(\inc ? 106 \then \Tmp{1}{102}) \\
    &= \set{t | t = \nil \lor \underline{(t_0 = \inc . x \land t' \in \Traces(\Tmp{1}{102})})} & \law{1.8.1}{L2} \\
    \Traces(\Tmp{1}{102})
    &= \Traces(\Tmp{1}{100} \intchoice \Tmp{2}{100}) \\
    &= \Traces(\Tmp{1}{100}) \union \underline{\Traces(\Tmp{2}{100})} & \law{3.2.3}{L1} \\
    \Traces(\Tmp{2}{100})
    &= \Traces(\inc ? 106 \then ((\outc ! 100 \then \Tmp{1}{102}) \intchoice \Stop) \\
    &\hspace{4em} \barchoice \outc ! 100 \then \inc ? 106 \then \Tmp{1}{102})) \\
    &= \set{t | t = \nil \lor \underline{(t_0 = \inc . 106 \land t' = \Traces((\outc ! 100 \then \Tmp{1}{102}) \intchoice \Stop))} \\
    &\hspace{5em}\lor (t_0 = \outc . 100 \land t' = \Traces(\inc ? 106 \then \Tmp{1}{102}))} & \law{1.8.1}{L3} \\
    \Traces((\outc ! 100 &\then \Tmp{1}{102}) \intchoice \Stop)
    = \Traces((\outc ! 100 \then \Tmp{1}{102})) \union \underline{\Traces(\Stop)} & \law{3.2.3}{L1} \\
    \Traces(\Stop)
    &= \set{\underline{\nil}} & \law{1.8.1}{L1}
\end{align*}
Notice that we here have substituted the variables with actual values, as this means we easily can choose the correct side of the \textit{if-else}-clause. We now show that $\Mill \after s = \Stop$.
\begin{align*}
    \Mill \after s
    &= (\inc ? x \then \Tmp{1}{x-4}) \after \trace{\inc . 106, \inc . 106} \\
    &= \Tmp{1}{102} \after \trace{\inc . 106} & \law{1.8.3}{L3A} \\
    &= (\Tmp{1}{100} \after \trace{\inc . 106}) \intchoice (\Tmp{2}{100} \after \trace{\inc . 106}) & \law{3.2.3}{L2} \\
    &= ((\outc ! 100 \then \Tmp{1}{102}) \after \nil) \intchoice (\Tmp{2}{100} \after \trace{\inc . 106}) & \law{1.8.3}{L3} \\
    &= (\outc ! 100 \then \Tmp{1}{102}) \intchoice ((\outc ! 100 \then \Tmp{1}{102}) \intchoice \Stop) \after \nil) & \law{1.8.3}{L3A} \\
    &= (\outc ! 100 \then \Tmp{1}{102}) \intchoice \Stop & \law{3.2.1}{L1}
\end{align*}
Now depending on which side of the internal choice $\Mill$ chooses we get that $\Mill \after s = \Stop$.

To see that we indeed have a deadlock we consider the state of $\Prod$ and $\Qual$. After the first action $\Qual$ will be trying to send $\rej$ as the log was too long. However as only $\Prod$ has the $\inc$-channel it is allowed to choose both of its branches, and thus it might choose the first as this does not require participation from the environment ($\Qual$). In this situtation $\Prod$ will be waiting on $\chk$ which $\Qual$ is unwilling to do at the moment, and $\Qual$ will be waiting on $\rej$ which $\Prod$ is unwilling to do. Thus we clearly have a deadlock.

% subsection b_ (end)

\subsection{c)} % (fold)
If we consider the trace from the previous question we could simply add one more action such that $t = \trace{\inc . 106, \inc . 106, \outc . 100}$. We have already calculated $\Mill \after s$. Thus we can just calculate $\Refusals(\Mill \after s)$. We first use \law{3.4.1}{L4}. Then since $\Stop$ refuses the entire alphabet, per \law{3.4.1}{L1}, we have that $\set{\outc . 100} \in \Refusals(\Mill \after s)$. We however also have that $\trace{\outc . 100} \in \Traces(\Mill \after s)$, as per \law{3.2.3}{L2} and \law{1.8.3}{L3A}. This proves that that $\Mill \after s$, and therefore also $\Mill$, may behave nondeterministically.
% subsection c_ (end)

% section problem_1 (end)

\section{Problem 2} % (fold)
\label{sec:problem_2}

\subsection{a)} % (fold)

We first show that $\Prod' \sqsubseteq_T \Prod$. We consider the composition of traces from $\Prod$, which has two different branches one can go down. Using \law{1.8.1}{L2-L3}, we can easily see that the first branch has the traces
\[
\set{\nil, \trace{\inc . x}, \trace{\inc . x, \chk . (x-4)}}
\]
In the same way the second branch has the traces
\[
\set{\nil, \trace{\rej . y}, \trace{\rej . y, \chk . (y-2)}}
\]
After the longest trace in both of the branches go back to $\Prod$ and can then again do either of the branches. Thus if we define the sets
\begin{align*}
S &= \set{\trace{\inc . x, \chk . (x-4)}, \trace{\rej . y, \chk . (y-2)}} \\
T &= \set{\nil, \trace{\inc . x}, \trace{\rej . y}}
\end{align*}
We can describe all traces of $\Prod$ as
\[
    \Traces(\Prod) = \set{(s^n) \cat t | n \in \mathbb{Z}_{\ge 0} \land s \in S \land t \in T}
\]
We now need to show that $\Prod'$ can do all of these traces too. We first rewrite the definition of $\Prod'$ by expanding $\Prodnx$ once and eliminating general choice
\begin{align*}
    \Prod'
    &= (\inc ? x \then \Prodnx) \extchoice (\rej ? y \then \chk ! (y-2) \then \Prod') \\
    & = (\inc ? x \then ((\chk ! (x-4) \then \Prod') \extchoice (\rej ? y \then \chk ! (y-2) \then \Prodnx))) \\
    &\quad \extchoice (\rej ? y \then \chk ! (y-2) \then \Prod') \\
    & = (\inc ? x \then ((\chk ! (x-4) \then \Prod') \barchoice (\rej ? y \then \chk ! (y-2) \then \Prodnx))) \\
    &\quad \barchoice (\rej ? y \then \chk ! (y-2) \then \Prod') & \law{3.3.1}{L5} \\
\end{align*}
If we ignore the inner branch which leads to $\Prodnx$ this definition has the exact same branches as $\Prod$. Thus we can go down either of these two branches to generate the same traces as in $\Prod$.

We then show that $\Prod \nsqsubseteq_T \Prod'$. This is done by exhibiting a trace $s \in \Traces(\Prod') \land s \notin \Traces(\Prod)$. Consider e.g. $s = \trace{\inc . x, \rej . y}$. We first try to produce the trace in $\Prod$
\begin{align*}
  \Traces(\Prod)
  &= \Traces((\inc ? x \then \chk ! (x-4) \then \Prod) \\
  &\quad\barchoice (\rej ? y \then \chk ! (y-2) \then \Prod)) \\
  &= \set{t | t = \nil \lor \underline{(t_0 = \inc . x \land t' \in \Traces(\chk ! (x-4) \then \Prod)}) \\
  &\hspace{5em} \lor (t_0 = \rej . y \land t' \in \Traces(\chk ! (y-2) \then \Prod))} & \law{1.8.1}{L3} \\
  \Traces(\chk ! (x-4) \then \Prod)
  &= \set{t | t = \nil \lor (t_0 = \chk . (x-4) \land t' \in \Traces(\Prod))} & \law{1.8.1}{L2}
\end{align*}
As can be seen $s \notin \Traces(\Prod)$. We now try to produce the trace in $\Prod'$
\begin{align*}
    \Traces(\Prod')
    &= \Traces(\inc ? x \then \Prodnx \barchoice \rej ? y \then \chk ! (y-2) \then \Prod') \\
    &= \set{t | t = \nil \lor \underline{(t_0 = \inc . x \land t' \in \Traces(\Prodnx)}) \\
    &\hspace{5em} \lor (t_0 = \rej . y \land t' \in \Traces(\chk ! (y-2) \then \Prod'))} & \law{1.8.1}{L3} \\
    \Traces(\Prodnx)
    &= \Traces(\chk ! (x-4) \then \Prod' \barchoice \rej ? y \then \chk ! (y-2) \then \Prodnx) \\
    &= \set{t | t = \nil \lor (t_0 = \chk . (x - 4) \land t' \in \Traces(\Prod')) \\
    &\hspace{5em} \lor \underline{(t_0 = \rej . y \land t' \in \Traces(\chk ! (y-2) \then \Prodnx)})} & \law{1.8.1}{L3} \\
    \Traces(\chk ! (y-2) \then \Prodnx)
    &= \set{t | \underline{t = \nil} \lor (t_0 = \chk . (y-2) \land t' \in \Traces(\Prodnx))} & \law{1.8.1}{L2}
\end{align*}
As seen $s \in \Traces(\Prod')$, proving that $\Prod \nsqsubseteq_T \Prod'$.

% subsection a_ (end)

\subsection{b)} % (fold)
The problem in 1b was that $\Prod$ was only able to do $\chk$ after it had done $\inc$, and since meant that we risked getting in a situation where a log is fails the tests, and thus $\Qual$ wishes to send it back on $\rej$. But instead of doing $\rej$ $\Prod$ could also $\inc$ again, meaning it would then be unwilling to do $\rej$ and instead wanting to do $\chk$, which $\Qual$ would be unwilling to do.

Thus the situtation we need to check in 
% subsection b_ (end)

\subsection{c)} % (fold)

% subsection c_ (end)

% section problem_2 (end)

\section{Problem 3} % (fold)
\label{sec:problem_3}

\subsection{a)} % (fold)
We first show that $\Prod'' \sqsubseteq_T \Prod'$. We consider the state of $\Prod'$ after some trace. Before reaching a process definition $\Prod'$ can do the trace $s = \trace{\inc . x}$ or $t = \trace{\rej . y, \chk . (y - 2)}$. after the traces we have (using \law{3.3.1}{L5} to eliminate external choice and then \law{1.8.3}{L3} to determine the process after the trace)
\[
    \Prod' \after s = \Prodnx \\
    \Prod' \after t = \Prod'
\]
We thus see, that if we do that latter trace in $\Prod''$, and $\Prod'' \after t = \Prod''$ Then $\Prod''$ can do every possible trace that $\Prod'$ can do when going down its latter branch (as the both reach their own definition after the trace). It is easily seen using \law{3.6.2}{L3} and \law{1.8.3}{L3A} that the trace $t$ results in $\Prodr$ being advanced up to its recursive definition and thus $\Prod'' \after t = \Prod''$.

We now consider the first branch of $\Prod'$. After doing the trace $s$ we reach $\Prodnx$, whereas we in $\Prod''$, using \law{3.6.2}{L3} and \law{1.8.3}{L3A}, have that
\[
    \Prod'' \after s = (\chk ! (x-4) \then \Prodi) \interleave \Prodr = \Prodnx''
\]
For ease of use we have renamed the process after $s$. Both processes can thus do the trace of the first branch, after which they reach a new process definition. We therefore now consider whether we can do the same traces in $\Prodnx''$ as can be done in $\Prodnx$.

As with $\Prod'$ we consider the traces of $\Prodnx$. Before reaching a process definition $\Prodnx$ can do the trace $p = \trace{\chk . (x-4)}$ or $t$. Again we have after each trace that (using \law{3.3.1}{L5} to eliminate external choice and then \law{1.8.3}{L3} to determine the process after the trace)
\[
    \Prodnx \after p = \Prod' \\
    \Prodnx \after t = \Prodnx
\]
Like before we then consider $\Prodnx'' \after t$. As before and citing the same laws $t$ advances $\Prodr$ up to its recursive definition, and thus
\[
    \Prodnx'' \after t = \Prodnx''
\]
As before $\Prodnx''$ can thus produce the same traces as the second branch of $\Prodnx$ because they both reach their own definition.

We then consider $\Prodnx'' \after p$. Again using using \law{3.6.2}{L3} and \law{1.8.3}{L3A}, we have that
\[
    \Prodnx'' \after p = \Prod''
\]

Thus we are now back at the original definition for both processes. This means that $\Prodnx''$ can do the traces of $\Prodnx$ if $\Prod''$ can do the traces of $\Prod'$, and that $\Prod''$ can do the traces of $\Prod'$ if $\Prodnx''$ can do the traces of $\Prodnx$. Since we know that second branches can be done, we have recursively covered all branches of $\Prod'$, meaning that $\Prod'' \sqsubseteq_T \Prod'$.

We now show that $\Prod' \nsqsubseteq_T \Prod''$. This is done by exhibiting a trace $s \in \Traces(\Prod'') \land s \notin \Traces(\Prod')$. Consider e.g. $s = \trace{\rej . y, \inc . x}$. We first try to produce the trace in $\Prod'$
\begin{align*}
    \Traces(\Prod')
    &= \Traces(\inc ? x \then \Prodnx \barchoice \rej ? y \then \chk ! (y-2) \then \Prod') \\
    &= \set{t | t = \nil \lor (t_0 = \inc . x \land t' \in \Traces(\Prodnx)) \\
    &\hspace{5em} \lor \underline{(t_0 = \rej . y \land t' \in \Traces(\chk ! (y-2) \then \Prod'))}} & \law{1.8.1}{L3} \\
    \Traces(\chk ! (y-2) \then \Prod')
    &= \set{t | t = \nil \lor (t_0 = \chk . (y-2) \land t' \in \Traces(\Prod'))} & \law{1.8.1}{L2}
\end{align*}
As can be seen we have no way to produce $s$ and so $s \notin \Traces(\Prod')$. We now try to produce the trace in $\Prod''$.
\begin{align*}
    \Traces(\Prod'')
    &= \Traces(\Prodi \interleave \Prodr) \\
    &= \set{s | \exists t : \Traces(\Prodi) \spot \exists u : \Traces(\Prodr) \spot s \Interleaves (t,u)}
\end{align*}
If we now define the traces $t = \trace{\inc . x}$ and $u = \trace{\rej . y}$, we need only show that $t \in \Traces(\Prodi)$ and $u \in \Traces(\Prodr)$, as one interleaving of these are $s$
\begin{align*}
    \Traces(\Prodr)
    &= \Traces(\inc ? x \then \chk ! (x-4) \then \Prodi) \\
    &= \set{t | t = \nil \lor \underline{(t_0 = \inc . x \land t' \in \Traces(\chk ! (x-4) \then \Prodi))}} & \law{1.8.1}{L2} \\
    \Traces(\chk ! (x-4) \then \Prodi)
    &= \set{t | \underline{t = \nil} \lor (t_0 = \chk . (x-4) \land t' \in \Traces(\Prodi))} & \law{1.8.1}{L2} \\ \\
    \Traces(\Prodr)
    &= \Traces(\rej ? y \then \chk ! (y-2) \then \Prodr) \\
    &= \set{t | t = \nil \lor \underline{(t_0 = \rej . y \land t' \in \Traces(\chk ! (y-2) \then \Prodr))}} & \law{1.8.1}{L2} \\
    \Traces(\chk ! (y-2) \then \Prodr)
    &= \set{t | \underline{t = \nil} \lor (t_0 = \chk . (y-2) \land t' \in \Traces(\Prodr))} & \law{1.8.1}{L2} \\ \\
\end{align*}
Thus we have shown that $s \in \Traces(\Prod'')$ and therefore that $\Prod' \nsqsubseteq_T$ \Prod''.
% subsection a_ (end)

\subsection{b)} % (fold)

We consider a trace where we read a value which is rejected first time around thrice (the third does not have to reject). This could e.g. be
\[
    t = \trace{\inc . 108, \inc . 108, \inc . 108}    
\]
To show that this can create a deadlock we consider the state of $\Qual$ and $\Prod''$ after the trace. For $\Prod''$ the trace we use will have $\rej$ and $\chk$ injected and the trace then becomes
\[
    t_1 = \trace{\inc . 108, \chk . 104, \rej . 104, \inc . 108, \chk . 104, \inc . 108}
\]
For $\Qual$ we remove $\inc$ and inject $\rej$ and $\chk$, giving us the following trace
\[
    t_2 = \trace{\chk . 104, \rej . 104, \chk . 104}
\]
We now run the two processes in parallel to show that we reach a point after $t$ where $\Mill''$ will be in a state of deadlock.
\begin{align*}
    \Prod' \after t_1
    &= ((\inc ? x \then \chk ! (x-4) \then \Prodi) \\
    &\quad\interleave (\rej ? y \then \chk ! (y-2) \then \Prodr)) \after t_1 \\
    \Qual \after t_2
    &= (\chk ? z \then ((\outc ! z \then \Qual) \\
    &\quad\lhd ok(z) \rhd (\rej ! z \then \Qual))) \after t_2
    \intertext{$\Prod'$ first read from $\inc$}
    \Prod' \after t_1
    &= ((\chk ! (104) \then \Prodi) \\
    &\quad \interleave (\rej ? y \then \chk ! (y-2) \then \Prodr)) \after t_1' & \law{3.6.2}{L3} + \law{1.8.3}{L3A} \\
    \Qual \after t_2
    &= (\chk ? z \then ((\outc ! z \then \Qual) \\
    &\quad \lhd ok(z) \rhd (\rej ! z \then \Qual))) \after t_2
    \intertext{Both are now willing to do $\chk$}
    \Prod' \after t_1
    &= (\Prodi \\
    &\quad\interleave (\rej ? y \then \chk ! (y-2) \then \Prodr)) \after t_1'' & \law{3.6.2}{L3} + \law{1.8.3}{L3A} \\
    \Qual \after t_2
    &= ((\outc ! 104 \then \Qual) \\
    &\quad \lhd ok(104) \rhd (\rej ! 104 \then \Qual)) \after t_2' & \law{1.8.3 }{L3A}
    \intertext{$ok(104)$ is false and, both are now willing to do $\rej$}
    \Prod' \after t_1
    &= ((\inc ? x \then \chk ! (x-4) \then \Prodi) \\
    &\quad\interleave (\chk ! (102) \then \Prodr)) \after \trace{\inc . 108, \chk . 104, \inc . 108} & \law{3.6.2}{L3} + \law{1.8.3}{L3A} \\
    \Qual \after t_2
    &= (\Qual) \after \trace{\chk . 104} & \law{5.5.1}{L8}+ \law{1.8.3 }{L3A}
    \intertext{$\Prod'$ now does another read from $\inc$}
    \Prod' \after t_1
    &= ((\chk ! (104) \then \Prodi) \\
    &\quad\interleave (\chk ! (102) \then \Prodr)) \after \trace{\chk . 104, \inc . 108} & \law{3.6.2}{L3} + \law{1.8.3}{L3A} \\
    \Qual \after t_2
    &= (\chk ? z \then ((\outc ! z \then \Qual) \\
    &\quad\lhd ok(z) \rhd (\rej ! z \then \Qual))) \after \trace{\chk . 104}
    \intertext{This is the turning point. Instead of handling the rejected log first we let $\Prodi$ do $\chk$ together with $\Qual$}
    \Prod' \after t_1
    &= (\Prodi \\
    &\quad\interleave (\chk ! (102) \then \Prodr)) \after \trace{\inc . 108} & \law{3.6.2}{L3} + \law{1.8.3}{L3A} \\
    \Qual \after t_2
    &= ((\outc ! 104 \then \Qual) \\
    &\quad\lhd ok(104) \rhd (\rej ! 104 \then \Qual)) & \law{1.8.3}{L3A}
    \intertext{Finally $\Prod'$ reads another value from $\inc$ and the system then looks as following}
    \Prod' \after t_1
    &= (\chk ! (104) \then \Prodi) \interleave (\chk ! (102) \then \Prodr) & \law{3.6.2}{L3} + \law{1.8.3}{L3A} \\
    \Qual \after t_2
    &= \rej ! 104 \then \Qual & \law{5.5.1}{L8}
\end{align*}
Clearly we have now reached a deadlock as both of $\Prodi$ and $\Prodr$ wants to send on $\chk$, which $\Qual$ is unwilling to receive, and $\Qual$ wants to send on $\rej$ which both $\Prodi$ and $\Prodr$ is unwilling to receive.
% subsection b_ (end)

\subsection{c)} % (fold)
To show this we first rewrite $\Prod''$
\begin{align*}
    \Prod''
    &= \Prodi \interleave \Prodr \\
    &= (\inc ? x \then ((\chk ! (x-4) \then \Prodi) \interleave \Prodr)) \\
    &\quad \extchoice (\rej ? y \then (\Prodi \interleave (\chk ! (y-2) \then \Prodr))) & \law{3.6.1}{L6}
    \intertext{We now define a new process $\Proda$}
    \Proda
    &= (\chk ! (x-4) \then \Prodi) \interleave \Prodr \\
    &= (\chk ! (x-4) \then (\Prodi \interleave \Prodr)) \\
    &\quad \extchoice (\rej ? y \then ((\chk ! (x-4) \then \Prodi) \interleave (\chk ! (y-2) \then \Prodr))) & \law{3.6.1}{L6} \\
    &= (\chk ! (x-4) \then \Prod'') \\
    &\quad \extchoice (\rej ? y \then ((\chk ! (x-4) \then \Prodi) \interleave (\chk ! (y-2) \then \Prodr)))
    \intertext{We now define a new process $\Prodb$}
    \Prodb
    &= \Prodi \interleave (\chk ! (y-2) \then \Prodr) \\
    &= (\inc ? x \then ((\chk ! (x-4) \then \Prodi) \interleave (\chk ! (y-2) \then \Prodr))) \\
    &\quad \extchoice (\chk ! (y-2) \then (\Prodi \interleave \Prodr)) & \law{3.6.1}{L6} \\
    &= (\inc ? x \then ((\chk ! (x-4) \then \Prodi) \interleave (\chk ! (y-2) \then \Prodr))) \\
    &\quad \extchoice (\chk ! (y-2) \then \Prod'')
    \intertext{We now define a new process $\Prodc$}
    \Prodc 
    &= (\chk ! (x-4) \then \Prodi) \interleave (\chk ! (y-2) \then \Prodr) \\
    &= (\chk ! (x-4) \then (\Prodi \interleave (\chk ! (y-2) \then \Prodr))) \\
    &\quad \extchoice (\chk ! (y-2) \then ((\chk ! (x-4) \then \Prodi) \interleave \Prodr)) & \law{3.6.1}{L6} \\
    &= (\chk ! (x-4) \then \Prodb \extchoice (\chk ! (y-2) \then \Proda) \\
    \intertext{Using the definition of each process we then have}
    \Prod''
    &= (\inc ? x \then \Proda \barchoice \rej ? y \then \Prodb) & \law{3.3.1}{L5}\\
    \Proda
    &= (\chk ! (x-4) \then \Prod'' \barchoice \rej ? y \then \Prodc) & \law{3.3.1}{L5} \\
    \Prodb
    &= (\inc ? x \then \Prodc \barchoice \chk ! (y-2) \then \Prod'') & \law{3.3.1}{L5} \\
    \Prodc 
    &= (\chk ! (x-4) \then \Prodb \extchoice (\chk ! (y-2) \then \Proda) \\
\end{align*}
% subsection c_ (end)

% section problem_3 (end)

\end{document}