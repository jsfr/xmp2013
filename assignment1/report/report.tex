\documentclass[a4paper, 11pt]{article}

\usepackage[utf8x]{inputenc}
\usepackage[T1]{fontenc}
\usepackage{ucs}
\usepackage[english]{babel}
\usepackage{mathtools, amsmath, amsfonts}
\usepackage{fancyhdr}
\usepackage[parfill]{parskip}
\usepackage{graphicx}
% \usepackage{palatino}
\usepackage{float}
\usepackage[font={small,it}]{caption}

% \linespread{1.05}
\pagestyle{fancyplain}
\fancyhead{}
\fancyfoot[L]{}
\fancyfoot[C]{}
\fancyfoot[R]{\thepage}
\renewcommand{\headrulewidth}{0pt}
\renewcommand{\footrulewidth}{0pt}
\setlength{\headheight}{13.6pt}

\widowpenalty=1000
\clubpenalty=1000

\newcommand{\horrule}[1]{\rule{\linewidth}{#1}}

\title{ 
\normalfont \normalsize 
\textsc{University of Copenhagen} \\ [25pt]
\horrule{0.5pt} \\[0.4cm]
\huge XMP: Assignment 1 \\
\horrule{2pt} \\[0.5cm]
}

\author{Jens Fredskov (chw752)}

\begin{document}
\maketitle

\part{Theory} % (fold)
\label{prt:theory_}

\section{} % (fold)

% section  (end)

\section{} % (fold)

\paragraph{a.} % (fold)
\label{par:a_}

The alphabet of the resulting process is $\alpha Sa = \{coin, pepsi, sprite, think\}$.
\begin{align*}
Sa &= VM || RICH \\
   &= (coin \rightarrow (pepsi \rightarrow VM | coin \rightarrow sprite \rightarrow VM)) \\
   &\phantom{=} || (coin \rightarrow think \rightarrow RICH) \\
   &= coin \rightarrow ((pepsi \rightarrow VM | coin \rightarrow sprite \rightarrow VM)) \\
   &\phantom{=} || (think \rightarrow RICH)) & [2.3.1\;L4A] \\
   &= coin \rightarrow ((pepsi \rightarrow (VM || think \rightarrow RICH)) \\
   &\phantom{=} | (think \rightarrow ((pepsi \rightarrow VM | coin \rightarrow sprite \rightarrow VM) || RICH))) & [2.3.1\;L7] \\
\end{align*}

We then look at
\begin{align*}
VM || think \rightarrow RICH &= think \rightarrow (VM || RICH) & [2.3.1\;L5B]\\
                             &=  think \rightarrow Sa
\end{align*}

We then look at
\begin{align*}
&(pepsi \rightarrow VM | coin \rightarrow sprite \rightarrow VM) || RICH \\
= &\phantom{(} pepsi \rightarrow Sa | coin \rightarrow ((sprite \rightarrow VM) || (think \rightarrow RICH)) & [2.3.1\;L7] \\
= &\phantom{(} pepsi \rightarrow Sa | coin \rightarrow (sprite \rightarrow think \rightarrow Sa | think \rightarrow sprite \rightarrow Sa) & [2.3.1\;L7]
\end{align*}

Which gives us the final result
\begin{align*}
Sa = coin \rightarrow (&pepsi \rightarrow think \rightarrow Sa | \\
    &think \rightarrow (pepsi \rightarrow Sa | coin \rightarrow (sprite \rightarrow think \rightarrow Sa | think \rightarrow sprite \rightarrow Sa)))
\end{align*}

% paragraph a_ (end)

\paragraph{b.} % (fold)
\label{par:b_}

The alphabet of the resulting process is $\alpha Sb = \{coin, pepsi, sprite, think\}$.
\begin{align*}
Sb &= Sa || THIRSTY1 \\
    &= coin \rightarrow (think \rightarrow (pepsi \rightarrow Sa \\
    &\phantom{=} | coin \rightarrow (sprite \rightarrow think \rightarrow Sa | think \rightarrow sprite \rightarrow Sa)) || THIRSTY1) & [2.3.1\;L7] \\
    &= coin \rightarrow think \rightarrow coin \rightarrow ((sprite \rightarrow think \rightarrow Sa | think \rightarrow sprite \rightarrow Sa) \\
    &\phantom{=} || THIRSTY1) & [2.3.1\;L5A\;\&\;L7] \\
    &= coin \rightarrow think \rightarrow coin \rightarrow (sprite \rightarrow think \rightarrow Sb | think \rightarrow sprite \rightarrow Sb) \\
\end{align*}

% paragraph b_ (end)

\paragraph{c.} % (fold)
\label{par:c_}

The alphabet of the resulting process is $\alpha Sc = \{coin, think, sprite, pepsi\}$. We first define:
\begin{align*}
Sc' &= (think \rightarrow RICH) || THIRSTY1 = think \rightarrow Sc | sprite \rightarrow Sc' & [2.3.1\;L7]
\end{align*}

We then find $Sc$ as
\begin{align*}
Sc &= (coin \rightarrow think \rightarrow RICH) || (sprite \rightarrow THIRSTY1) \\
   &= coin \rightarrow ((think \rightarrow RICH) || THIRSTY1) \\
   &\phantom{=} | sprite \rightarrow (RICH || THIRSTY1) & [2.3.1\;L7] \\
   &= coin \rightarrow Sc' | sprite \rightarrow Sc 
\end{align*}

% paragraph c_ (end)

\paragraph{d.} % (fold)
\label{par:d_}

The alphabet of the resulting process is $\alpha Sd = \{coin, think, sprite, pepsi \}$.
\begin{align*}
Sd &= (coin \rightarrow (pepsi \rightarrow VM | coin \rightarrow sprite \rightarrow VM)) \\
   &\phantom{=} || (coin \rightarrow Sc' | sprite \rightarrow Sc) \\
   &= coin \rightarrow (pepsi \rightarrow VM | coin \rightarrow sprite \rightarrow VM) || Sc' & [2.3.1\;L7] \\
   &= coin \rightarrow think \rightarrow (Sc || (pepsi \rightarrow VM | coin \rightarrow sprite \rightarrow VM)) & [2.3.1\;L7] \\
   &= coin \rightarrow think \rightarrow coin \rightarrow (Sc' || (sprite \rightarrow VM)) & [2.3.1\;L7] \\
   &= coin \rightarrow think \rightarrow coin \rightarrow (think \rightarrow (Sc || (sprite \rightarrow VM)) | sprite \rightarrow (Sc' || VM)) & [2.3.1\;L7] \\
   &= coin \rightarrow think \rightarrow coin \rightarrow (think \rightarrow sprite \rightarrow Sd | sprite \rightarrow think \rightarrow Sd) & [2 \times 2.3.1\;L7]
\end{align*}

% paragraph d_ (end)

\paragraph{e.} % (fold)
\label{par:e_}

The alphabet of the resulting process is $\alpha Se = \{ coin, pepsi, sprite, think \}$.
\begin{align*}
Se &= Sa || THIRSTY2 \\
   &= (coin \rightarrow (pepsi \rightarrow think \rightarrow Sa | \\
   &\phantom{=} think \rightarrow (pepsi \rightarrow Sa | coin \rightarrow (sprite \rightarrow think \rightarrow Sa \\
   &\phantom{=} | think \rightarrow sprite \rightarrow Sa)))) || THIRSTY2) \\
   &= coin \rightarrow (pepsi \rightarrow ((think \rightarrow Sa) || THIRSTY2) \\
   &\phantom{=} | think \rightarrow ((pepsi \rightarrow Sa | coin \rightarrow (sprite \rightarrow think \rightarrow Sa \\
   &\phantom{=} | think \rightarrow sprite \rightarrow Sa))) || THIRSTY2)) & [2.3.1\;L5A\;\&\;L7] \\
\end{align*}

We then calculate
\begin{align*}
(think -> Sa) &|| THIRSTY2 = think -> Se & [2.3.1\;L5A]
\end{align*}

We then calculate 
\begin{align*}
& ((pepsi \rightarrow Sa | coin \rightarrow (sprite \rightarrow think \rightarrow Sa \\
&\phantom{=} | think \rightarrow sprite \rightarrow Sa))) || THIRSTY2)) \\
&= pepsi \rightarrow Se | coin \rightarrow ((sprite \rightarrow think \rightarrow Sa \\
&\phantom{=} | think \rightarrow sprite \rightarrow Sa) || THIRSTY2) & [2.3.1\;L7] \\
&= pepsi \rightarrow Se | coin \rightarrow think \rightarrow ((sprite \rightarrow Sa) || TH2) & [2.3.1\;L7] \\
&= pepsi \rightarrow Se | coin \rightarrow think \rightarrow STOP & [2.3.1\;L7]
\end{align*}

The final result is then
\begin{align*}
Se = coin \rightarrow (pepsi \rightarrow think \rightarrow Se | think \rightarrow (pepsi \rightarrow Se | coin \rightarrow think \rightarrow STOP))
\end{align*}

% paragraph e_ (end)

\paragraph{f.} % (fold)
\label{par:f_}

The alphabet of the resulting process is $\alpha Sf = \{coin, think, sprite, pepsi\}$. We first calculate
\begin{align*}
f(Sc) = coin \rightarrow f(Sc') | pepsi \rightarrow f(Sc)
\end{align*}

\begin{align*}

\end{align*}

% paragraph f_ (end)

% section  (end)

\section{} % (fold)

\paragraph{a.} % (fold)
\label{par:a_}
2a-d does not include the $STOP$ process, and every branch ends with a recursive reference, thus no trace can reach a $STOP$ and deadlock.
% paragraph a_ (end)

\paragraph{b.} % (fold)
\label{par:b_}
Consider the following trace $<coin,think,coin,think,STOP>$. 
% paragraph b_ (end)

\paragraph{c.} % (fold)
\label{par:c_}

% paragraph c_ (end)

% section  (end)

% part theory_ (end)

\part{Programming} % (fold)
\label{prt:programming_}

\section{Implementation} % (fold)
\label{sec:implementation}

% section implementation (end)

% part programming_ (end)

\end{document}