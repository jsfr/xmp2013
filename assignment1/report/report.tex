\documentclass[a4paper, 11pt]{article}

\usepackage[utf8x]{inputenc}
\usepackage[T1]{fontenc}
\usepackage{ucs}
\usepackage[english]{babel}
\usepackage{mathtools, amsmath, amsfonts}
\usepackage{fancyhdr}
\usepackage[parfill]{parskip}
\usepackage{graphicx}
\usepackage{palatino,newtxmath}
\usepackage{float}
\usepackage[font={small,it}]{caption}
\usepackage{minted}

% \linespread{1.05}
\pagestyle{fancyplain}
\fancyhead{}
\fancyfoot[L]{}
\fancyfoot[C]{}
\fancyfoot[R]{\thepage}
\renewcommand{\headrulewidth}{0pt}
\renewcommand{\footrulewidth}{0pt}
\setlength{\headheight}{13.6pt}

\widowpenalty=1000
\clubpenalty=1000

\newcommand{\horrule}[1]{\rule{\linewidth}{#1}}
\def\sat{\mathbf{\;sat\;}}

\title{ 
\normalfont \normalsize 
\textsc{University of Copenhagen} \\ [25pt]
\horrule{0.5pt} \\[0.4cm]
\huge XMP: Assignment 1 \\
\horrule{2pt} \\[0.5cm]
}

\author{Jens Fredskov (chw752)}

\begin{document}
\maketitle
\pagebreak
\tableofcontents
\pagebreak

\part{Theory} % (fold)
\label{prt:theory_}

\section{} % (fold)

The definition of $VM$ can also be written as:
\[
    VM = \mu X : \{ coin, pepsi, sprite\} \bullet coin \rightarrow (pepsi \rightarrow X | coin \rightarrow sprite \rightarrow X)
\]

We first use [1.10.1 L6] and show that STOP satifies the specification
\begin{align*}
STOP & \sat (tr = \langle \rangle) & [1.10.1\;L4A] \\
     & \Rightarrow (tr \downarrow coin) - (tr \downarrow pepsi) - 2 \cdot (tr \downarrow sprite) = 0 \le 2 \\
     & \Rightarrow STOP \sat (tr \downarrow coin) - (tr \downarrow pepsi) - 2 \cdot (tr \downarrow sprite) \le 2 & [1.10.1\;L3]
\end{align*}

Now assume that $X \sat (tr \downarrow coin) - (tr \downarrow pepsi) - 2 \cdot (tr \downarrow sprite) \le 2$
\begin{align*}
coin & \rightarrow (pepsi \rightarrow X | coin \rightarrow sprite \rightarrow X) \\
     & \sat (tr = \langle \rangle \vee tr_0 = coin \wedge S(tr')) & [1.10.1\;L4B] \\
     & \Rightarrow (tr \downarrow coin) - (tr \downarrow pepsi) - 2 \cdot (tr \downarrow sprite) \\
     & = 1 + (tr' \downarrow coin) - (tr' \downarrow pepsi) - 2 \cdot (tr' \downarrow sprite) \le 2
\end{align*}

We then use [1.10.1 L4D]
\begin{align*}
& \text{if } pepsi \rightarrow X \sat ((tr \downarrow coin) - (tr \downarrow pepsi) - 2 \cdot (tr \downarrow sprite) \le 2) \\
& \text{and } coin \rightarrow sprite \rightarrow X \sat ((tr \downarrow coin) - (tr \downarrow pepsi) - 2 \cdot (tr \downarrow sprite) \le 2) \\
& \text{then } (pepsi \rightarrow X | coin \rightarrow sprite \rightarrow X) \sat (tr = \langle \rangle \vee ((tr_0 = pepsi \wedge S(X)) \vee (tr_0 = coin \wedge S(tr')))) \\
& \Rightarrow (tr \downarrow coin) - (tr \downarrow pepsi) - 2 \cdot (tr \downarrow sprite) \\
& = \pm 1 + (tr' \downarrow coin) - (tr' \downarrow pepsi) - 2 \cdot (tr' \downarrow sprite) \le 2
\end{align*}

Again we use [1.10.1 L4B]
\begin{align*}
sprite \rightarrow X &\sat (tr = \langle \rangle \vee (tr_0 = sprite \wedge S(X))) \\
& \Rightarrow (tr \downarrow coin) - (tr \downarrow pepsi) - 2 \cdot (tr \downarrow sprite) \\
& = -2 + (tr' \downarrow coin) - (tr' \downarrow pepsi) - 2 \cdot (tr' \downarrow sprite) \le 2
\end{align*}

We can now put all the parts together by [1.10.1 L6] and we have that
\[
    (tr \downarrow coin) - (tr \downarrow pepsi) - 2 \cdot (tr \downarrow sprite) = (1 + 1 - 2 = 0 \vee 1 + 1 - 0 = 2) \le 2
\]

% section  (end)

\section{} % (fold)

\paragraph{a.} % (fold)
\label{par:a_}

The alphabet of the resulting process is $\alpha Sa = \{coin, pepsi, sprite, think\}$.
\begin{align*}
Sa &= VM || RICH \\
   &= (coin \rightarrow (pepsi \rightarrow VM | coin \rightarrow sprite \rightarrow VM)) \\
   &\phantom{=} || (coin \rightarrow think \rightarrow RICH) \\
   &= coin \rightarrow ((pepsi \rightarrow VM | coin \rightarrow sprite \rightarrow VM)) \\
   &\phantom{=} || (think \rightarrow RICH)) & [2.3.1\;L4A] \\
   &= coin \rightarrow ((pepsi \rightarrow (VM || think \rightarrow RICH)) \\
   &\phantom{=} | (think \rightarrow ((pepsi \rightarrow VM | coin \rightarrow sprite \rightarrow VM) || RICH))) & [2.3.1\;L7] \\
\end{align*}

We then look at
\begin{align*}
VM || think \rightarrow RICH &= think \rightarrow (VM || RICH) & [2.3.1\;L5B]\\
                             &=  think \rightarrow Sa
\end{align*}

We then look at
\begin{align*}
&(pepsi \rightarrow VM | coin \rightarrow sprite \rightarrow VM) || RICH \\
= &\phantom{(} pepsi \rightarrow Sa | coin \rightarrow ((sprite \rightarrow VM) || (think \rightarrow RICH)) & [2.3.1\;L7] \\
= &\phantom{(} pepsi \rightarrow Sa | coin \rightarrow (sprite \rightarrow think \rightarrow Sa | think \rightarrow sprite \rightarrow Sa) & [2.3.1\;L7]
\end{align*}

Which gives us the final result
\begin{align*}
Sa = coin \rightarrow (&pepsi \rightarrow think \rightarrow Sa | \\
    &think \rightarrow (pepsi \rightarrow Sa | coin \rightarrow (sprite \rightarrow think \rightarrow Sa | think \rightarrow sprite \rightarrow Sa)))
\end{align*}

% paragraph a_ (end)

\paragraph{b.} % (fold)
\label{par:b_}

The alphabet of the resulting process is $\alpha Sb = \{coin, pepsi, sprite, think\}$.
\begin{align*}
Sb &= Sa || THIRSTY1 \\
    &= coin \rightarrow (think \rightarrow (pepsi \rightarrow Sa \\
    &\phantom{=} | coin \rightarrow (sprite \rightarrow think \rightarrow Sa | think \rightarrow sprite \rightarrow Sa)) || THIRSTY1) & [2.3.1\;L7] \\
    &= coin \rightarrow think \rightarrow coin \rightarrow ((sprite \rightarrow think \rightarrow Sa | think \rightarrow sprite \rightarrow Sa) \\
    &\phantom{=} || THIRSTY1) & [2.3.1\;L5A\;\&\;L7] \\
    &= coin \rightarrow think \rightarrow coin \rightarrow (sprite \rightarrow think \rightarrow Sb | think \rightarrow sprite \rightarrow Sb) \\
\end{align*}

% paragraph b_ (end)

\paragraph{c.} % (fold)
\label{par:c_}

The alphabet of the resulting process is $\alpha Sc = \{coin, think, sprite, pepsi\}$. We first define:
\begin{align*}
Sc' &= (think \rightarrow RICH) || THIRSTY1 = think \rightarrow Sc | sprite \rightarrow Sc' & [2.3.1\;L7]
\end{align*}

We then find $Sc$ as
\begin{align*}
Sc &= (coin \rightarrow think \rightarrow RICH) || (sprite \rightarrow THIRSTY1) \\
   &= coin \rightarrow ((think \rightarrow RICH) || THIRSTY1) \\
   &\phantom{=} | sprite \rightarrow (RICH || THIRSTY1) & [2.3.1\;L7] \\
   &= coin \rightarrow Sc' | sprite \rightarrow Sc 
\end{align*}

% paragraph c_ (end)

\paragraph{d.} % (fold)
\label{par:d_}

The alphabet of the resulting process is $\alpha Sd = \{coin, think, sprite, pepsi \}$.
\begin{align*}
Sd &= (coin \rightarrow (pepsi \rightarrow VM | coin \rightarrow sprite \rightarrow VM)) \\
   &\phantom{=} || (coin \rightarrow Sc' | sprite \rightarrow Sc) \\
   &= coin \rightarrow (pepsi \rightarrow VM | coin \rightarrow sprite \rightarrow VM) || Sc' & [2.3.1\;L7] \\
   &= coin \rightarrow think \rightarrow (Sc || (pepsi \rightarrow VM | coin \rightarrow sprite \rightarrow VM)) & [2.3.1\;L7] \\
   &= coin \rightarrow think \rightarrow coin \rightarrow (Sc' || (sprite \rightarrow VM)) & [2.3.1\;L7] \\
   &= coin \rightarrow think \rightarrow coin \rightarrow (think \rightarrow (Sc || (sprite \rightarrow VM)) | sprite \rightarrow (Sc' || VM)) & [2.3.1\;L7] \\
   &= coin \rightarrow think \rightarrow coin \rightarrow (think \rightarrow sprite \rightarrow Sd | sprite \rightarrow think \rightarrow Sd) & [2 \times 2.3\;L7]
\end{align*}

% paragraph d_ (end)

\paragraph{e.} % (fold)
\label{par:e_}

The alphabet of the resulting process is $\alpha Se = \{ coin, pepsi, sprite, think \}$.
\begin{align*}
Se &= Sa || THIRSTY2 \\
   &= (coin \rightarrow (pepsi \rightarrow think \rightarrow Sa | \\
   &\phantom{=} think \rightarrow (pepsi \rightarrow Sa | coin \rightarrow (sprite \rightarrow think \rightarrow Sa \\
   &\phantom{=} | think \rightarrow sprite \rightarrow Sa)))) || THIRSTY2) \\
   &= coin \rightarrow (pepsi \rightarrow ((think \rightarrow Sa) || THIRSTY2) \\
   &\phantom{=} | think \rightarrow ((pepsi \rightarrow Sa | coin \rightarrow (sprite \rightarrow think \rightarrow Sa \\
   &\phantom{=} | think \rightarrow sprite \rightarrow Sa))) || THIRSTY2)) & [2.3.1\;L5A\;\&\;L7] \\
\end{align*}

We then calculate
\begin{align*}
(think -> Sa) &|| THIRSTY2 = think -> Se & [2.3.1\;L5A]
\end{align*}

We then calculate 
\begin{align*}
& ((pepsi \rightarrow Sa | coin \rightarrow (sprite \rightarrow think \rightarrow Sa \\
&\phantom{=} | think \rightarrow sprite \rightarrow Sa))) || THIRSTY2)) \\
&= pepsi \rightarrow Se | coin \rightarrow ((sprite \rightarrow think \rightarrow Sa \\
&\phantom{=} | think \rightarrow sprite \rightarrow Sa) || THIRSTY2) & [2.3.1\;L7] \\
&= pepsi \rightarrow Se | coin \rightarrow think \rightarrow ((sprite \rightarrow Sa) || TH2) & [2.3.1\;L7] \\
&= pepsi \rightarrow Se | coin \rightarrow think \rightarrow STOP & [2.3.1\;L7]
\end{align*}

The final result is then
\begin{align*}
Se = coin \rightarrow (pepsi \rightarrow think \rightarrow Se | think \rightarrow (pepsi \rightarrow Se | coin \rightarrow think \rightarrow STOP))
\end{align*}

% paragraph e_ (end)

\paragraph{f.} % (fold)
\label{par:f_}

The alphabet of the resulting process is $\alpha Sf = \{coin, think, sprite, pepsi\}$. We first calculate
\begin{align*}
f(Sc) &= coin \rightarrow f(Sc') | pepsi \rightarrow f(Sc) \\
f(Sc') &= think \rightarrow f(Sc) | pepsi \rightarrow f(Sc')
\end{align*}

We then find $Sf$
\begin{align*}
Sf &= (coin \rightarrow (pepsi \rightarrow VM | coin \rightarrow sprite \rightarrow VM)) \\
   &\phantom{=} || (coin \rightarrow f(Sc') | pepsi \rightarrow f(Sc)) \\
   &= coin \rightarrow ( (pepsi \rightarrow VM | coin \rightarrow sprite \rightarrow VM) || (think \rightarrow f(Sc) | pepsi \rightarrow f(Sc')) ) & [2.3.1\;L7] \\
   &= coin \rightarrow (pepsi \rightarrow (VM || f(Sc')) | think \rightarrow ((pepsi \rightarrow VM \\
   &\phantom{=} | coin \rightarrow sprite \rightarrow VM) || f(Sc))) & [2.3.1\;L7]
\end{align*}

We then calculate
\begin{align*}
VM &|| f(Sc') = think \rightarrow Sf & [2.3.1\;L7]
\end{align*}

We then calculate
\begin{align*}
& (pepsi \rightarrow VM | coin \rightarrow sprite \rightarrow VM) || f(Sc) \\
&= pepsi \rightarrow Sf | coin \rightarrow ((sprite \rightarrow VM) || f(Sc')) & [2.3.1\;L7] \\
&= pepsi \rightarrow Sf | coin \rightarrow think -> STOP & [2\times2.3\;L7]
\end{align*}

The final result is then
\begin{align*}
Sf = coin \rightarrow (pepsi \rightarrow think \rightarrow Sf | think \rightarrow (pepsi \rightarrow Sf | coin \rightarrow think \rightarrow STOP))
\end{align*}

% paragraph f_ (end)

% section  (end)

\section{} % (fold)

\paragraph{a.} % (fold)
\label{par:a_}
2a-d does not include the $STOP$ process, and every branch ends with a recursive reference, thus no trace can reach a $STOP$ and deadlock.
% paragraph a_ (end)

\paragraph{b.} % (fold)
\label{par:b_}
Consider the following trace $s = \langle coin, think, coin, think \rangle$. This trace reaches is in $Se$ which can be seen by using the laws [1.8.1 L2], [1.8.1 L3], [1.8.1 L3], [1.8.1 L2] and [1.8.1 L1].

We then show that we reach $STOP$ after the trace
\begin{align*}
Se &= (coin \rightarrow (pepsi \rightarrow think \rightarrow Se \\
   &\phantom{=} | think \rightarrow (pepsi \rightarrow Se | coin \rightarrow think \rightarrow STOP))) / \langle coin, think, coin, think \rangle \\
   &= (pepsi \rightarrow think \rightarrow Se \\
   &\phantom{=} | think \rightarrow (pepsi \rightarrow Se | coin \rightarrow think \rightarrow STOP)) / \langle think, coin, think \rangle & [1.8.3\;L3A] \\
   &= (coin \rightarrow think \rightarrow STOP) / \langle coin, think \rangle & [1.8.3\;L3] \\
   &= (think \rightarrow STOP) / \langle think \rangle & [1.8.3\;L3A] \\
   &= STOP & [1.8.3\;L3A] 
\end{align*}

% paragraph b_ (end)

\paragraph{c.} % (fold)
\label{par:c_}

To show this we use the laws [2.2.3 L1] and [2.2.3 L2]. Firstly the trace $\langle coin \rangle$ is contained as it is the first action of both $VM$ and $RICH$, and is not in the alphabet of $THIRSTY2$. Secondly the trace $\langle coin, think \rangle$ is contained as $RICH$ now wants to do a $think$ and since it is not in the alphabet of $VM$ and $THIRSTY2$. Thirdly $\langle coin, think, coin \rangle$ is contained as $VM$ can do a coin in one of its branches, and $RICH$ is back to its original state. Finally $s = \langle coin, think, coin, think \rangle$ is contained as $RICH$ can once again do a $think$.

% paragraph c_ (end)

\paragraph{d.} % (fold)
\label{par:d_}

\begin{align*}
&(VM || RICH) || THIRSTY2 / s \\
&= 
((coin \rightarrow (pepsi \rightarrow VM | coin \rightarrow sprite \rightarrow VM) || (coin \rightarrow think \rightarrow RICH)) \\
&\phantom{=} || (pepsi \rightarrow THIRSTY2) / s \\
&= ((pepsi \rightarrow VM | coin \rightarrow sprite \rightarrow VM) || (think \rightarrow RICH)) \\
&\phantom{=} || (pepsi \rightarrow THIRSTY2) / \langle think, coin, think \rangle \\
&= ((pepsi \rightarrow VM | coin \rightarrow sprite \rightarrow VM) || RICH) \\
&\phantom{=} || (pepsi \rightarrow THIRSTY2) / \langle coin, think \rangle \\
&= (sprite \rightarrow VM) || (think \rightarrow RICH)) || (pepsi -> THIRSTY2) / \langle think \rangle \\
&= (sprite \rightarrow VM) || (coin \rightarrow think \rightarrow RICH)) || (pepsi \rightarrow THIRSTY2)
\end{align*}

We have now reached a deadlocked state as no possible event can happen. $coin$ is blocked by the first process as it has this in its alphabet but is not ready to do one, the same goes for $pepsi$. $think$ is blocked by the second process as it is not ready to do this, and finally $sprite$ is blocked by the third process which is only ready to do a $pepsi$, but have $sprite$ in its alphabet.

% paragraph d_ (end)

% section  (end)

% part theory_ (end)

\part{Programming} % (fold)
\label{prt:programming_}

\section{Implementation} % (fold)
\label{sec:implementation}

The solution has been implemented in the Go language. All source code can be found in Appendix \ref{sec:source_code}.

To run the program implementing the three problems, one must run the \texttt{main.go} main-function (this is done by running \texttt{go run main} in the source directory). This functions starts a machine for each problem (in serial order) runs it, receives its bin counts and prints these.

\subsection{Components} % (fold)
\label{sub:components}

\paragraph{The injector} % (fold)
\label{par:the_injector}

Simply runs through the number of balls specified and sends a message to the top pin for each ball. There is no buffer on the channel.

% paragraph the_injector (end)

% subsection components (end)

\subsection{Two Layer Machine} % (fold)
\label{sub:two_layer_machine}

% subsection two_layer_machine (end)

\subsection{N Layer Machine} % (fold)
\label{sub:n_layer_machine}

% subsection n_layer_machine (end)

\subsection{Oscillating Machine} % (fold)
\label{sub:oscillating_machine}

% subsection oscillating_machine (end)

% section implementation (end)

\appendix
\section{Source code} % (fold)
\label{sec:source_code}

\inputminted[fontsize=\scriptsize, frame=topline, label=machine.go, linenos=true]{go}{../src/machine.go}

\inputminted[fontsize=\scriptsize, frame=topline, label=main.go, linenos=true]{go}{../src/main.go}

% section source_code (end)

% part programming_ (end)

\end{document}